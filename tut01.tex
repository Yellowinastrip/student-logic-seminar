\documentclass{article}
\usepackage{babel}[english,czech]
\usepackage{inputenc}[utf8]
\usepackage{amsmath}
\usepackage{amsthm}
\usepackage{amssymb}
\usepackage{indentfirst}

\theoremstyle{definition}
\newtheorem{thrm}{Theorem}
\newtheorem{excs}[thrm]{Exercise}
\newtheorem{exmp}[thrm]{Example}
\newtheorem{prbl}[thrm]{Open problem}
\newtheorem{defi}[thrm]{Definition}
\newtheorem{defip}[thrm]{``Definition"}

\newcommand{\abs}[1]{\lvert #1 \rvert}
\renewcommand{\P}{\textbf{P}}
\newcommand{\NP}{\textbf{NP}}
\newcommand{\coNP}{\textbf{coNP}}
\newcommand{\Land}{\bigwedge}
\newcommand{\Lor}{\bigvee}

\begin{document}
\section*{Formal Proofs and their Lengths I}

\subsection*{Basic propositional logic}

\begin{defi}
    Let $V=\{x_1,x_2, x_3, \dots\}$ be a countable set, we will call $V$ the set of \emph{propositional variables} (atoms). We define a \emph{propositional formula} (in the DeMorgan Language) to be a word defined by the following recursive conditions:
    \begin{itemize}
        \item $A$ is a formula,  if it is a propositional variable.
        \item $A$ is a formula, if it is of the form $(B \land C)$, where $B$ and $C$ are formulas.
        \item $A$ is a formula, if it is of the form $(B \lor C)$, where $B$ and $C$ are formulas.
        \item $A$ is a formula, if it is of the form $\lnot B$, where $B$ is a formula.
    \end{itemize}
    A \emph{subformula} of a formula $A$ is a subword of $A$ which is also a formula. The notation $A(p_1,\dots,p_n)$ means that the propositional variables occuring in $A$ are among the set $\{p_1,\dots,p_n\}$.
\end{defi}

\begin{defi}
    Let $A(p_1,\dots,p_n)$ be a propositional formula. We call any function $h:\{p_1,\dots,p_n\}\to \{0,1\}$ a truth assignment (of $A$). Any truth assignment can be extended to give a $\{0,1\}$-value to $A$ by the obvious recursive definition. If $h(p_i)=b_i$ for each $1\leq i\leq n$, we denote the value $h(A)$ as $A(b_1,\dots,b_n)$. 
    
    We say $A$ is \emph{satisfiable} if there is a truth assignment such that $h(A)=1$, otherwise we call it \emph{unsatisfiable}.
    We say $A$ is a \emph{tautology} if every truth assignment $h$ results in $h(A)=1$. 
\end{defi}

\begin{excs}
    Observe that a propositional formula $A$ is a tautology iff $\lnot A$ is unsatisfiable.
\end{excs}

\begin{defi}
    A function of the form $f:\{0,1\}^n\to\{0,1\}$ is a \emph{Boolean function}, every propositional formula $A(p_1,\dots,p_n)$ determines the \emph{truth-table function} $\textbf{tt}_A$ as \[\textbf{tt}_A: (b_1,\dots,b_n)\mapsto A(b_1,\dots,b_n).\]
\end{defi}

\begin{excs}
    Show that every Boolean function is a truth-table function of some propositional formula $A$. 
\end{excs}

\begin{excs}
    Show that for every propositional Boolean formula in the De Morgan language $A$ there exists a formula\footnote{This is not a propositional formula by our definition, but you can check an analogous definition can be made for this set of connectives.} $A'$ in the language using only the connectives form the set $\{\lnot, \to\}$ (interpreted as negation and implication) such that $\textbf{tt}_A = \textbf{tt}_{A'}$.
\end{excs}

\begin{defi}
    A propositional formula $A$ is in the conjunctive normal form (CNF) if it is of the form $\Land_i \Lor_j \ell_{ij}$, where each $\ell_{ij}$ is either a propositional variable or a negation of one (a literal).

    A propositional formula $A$ is in the disjunctive normal form (DNF) if it is of the form $\Lor_i \Land_j \ell_{ij}$, where each $\ell_{ij}$ is a literal.

    Disjunctions of literals are called \emph{clauses}, and conjunctions of literals are called \emph{logical terms}.
\end{defi}

\begin{excs}
    Show that every Boolean function is a truth-table function of some DNF $A$ and some CNF $B$.
\end{excs}

\begin{excs}
    Show there is a fast (polynomial time) algorithm deciding whether a DNF $A$ is satisfiable.
\end{excs}

\begin{excs}
    Show there is a fast (polynomial time) algorithm deciding whether a CNF $A$ is a tautology.
\end{excs}

\begin{excs}
    Show that there is a Boolean function such that its smallest DNF representation is exponentially smaller than its CNF representation (or vice-versa).
\end{excs}


\begin{excs}[bonus] 
    Show that for each polynomial $p(x)$ there is a Boolean function with $n$ inputs, which is not a truth-table function of any propositional formual $A$ with less than $p(n)$ symbols.
\end{excs}

\subsection*{Propositional Proof Systems}

\begin{defi}
    Let $A$ be a finite set of symbols. We define $A^{\leq n}:= \bigcup_{i=0}^n A^i$ and $A^* := \bigcup_{i\geq 0} A^i$.
\end{defi}

\begin{defi}
    A predicate $f:\{0,1\}^* \to \{0,1\}$ is in $\P$ if there is a Turing machine $M$ computing $f$ in polynomial time\footnote{The precise definition of a Turing machine in fact does not matter. If you have never encountered the definition of a Turing machine, it is enough to consider the intuitive idea of an algorithm, whose number of steps does not exceed a specific polynomial in the length of the input and this itself just means, that the algorithm is somehow feasible --- does not run too long. For example, such an algorithm cannot look at every truth assignment of a formula it receives as an input.}.
\end{defi}

\begin{defi}[Cook-Reckhow] A \emph{propositional proof system} (or a PPS) $P$ is determined by a predicate $f(x,y)$ in $\P$ such that for every propositional formula $A$:
    \[\text{$A$ is a tautology} \iff (\exists y\in \{0,1\}^*) \, f(A,y),\]
    here we interpret $f$ to be a predicate checking that $y$ is a valid ``proof'' of $A$. That is, if $f(A,y)=1$, then we say $y$ is a $P$-proof of $A$.
\end{defi}

\begin{exmp}
    The truth-table proof system is a system determined by a predicate     
    \[f(A,y)=\begin{cases}
        1 & \text{$y$ is the truth-table of $A$, $(\forall \overline x) \textbf{tt}_A(\overline x)=1$,}\\
        0 & \text{otherwise.}
    \end{cases}\]
\end{exmp}

\begin{excs}
    Show that the truth-table proof system is a propositional proof system by the definition of Cook-Reckhow.
\end{excs}

\begin{excs}[First lower bound!]
    Show that every truth-table proof of a tautology is exponentially long in the size of that tautology.
\end{excs}


\subsection*{A Little Bit of Complexity}

\begin{defi}[*]
    A predicate $f: \{0,1\}^* \to \{0,1\}$ is in $\NP$ if there is a function $g(x,y)$ in $\P$ and a polynomial $p$ such that for every $x\in \{0,1\}^n$:
    \[ f(x)=1 \iff (\exists y\in\{0,1\}^{\leq p(n)}) \, g(x,y)=1, \]
    if such a $y$ exists it is called the \emph{witness}.
\end{defi}

\begin{defi}[*]
    A predicate $f: \{0,1\}^* \to \{0,1\}$ is in $\coNP$ if there is a function $g(x,y)$ in $\P$ and a polynomial $p$ such that for every $x\in \{0,1\}^n$:
    \[ f(x)=0 \iff (\exists y\in\{0,1\}^{\leq p(n)}) \, g(x,y)=0.\]
\end{defi}


\begin{excs}[*]
    Show that $f(x)\in \NP$ if and only if $\lnot f(x) \in \coNP$. 
\end{excs}

\begin{thrm}[Cook-Reckhow]
    $\NP=\coNP$ if and only if there is a propositional proof system $P$ which has polynomial sized $P$-proofs of every tautology.
\end{thrm}

\begin{excs}[*]
    Prove the Cook-Reckhow theorem.
\end{excs}

\subsection*{Frege systems I}

\begin{defi}
    The textbook Frege proof system is determined by the proofs of the following form:

    The connectives in every formula in the system are just $\{\lnot, \to \}$. A proof of a formula $A$ is a sequence of propositional formulas $(B_1,\dots, B_k)$, where $B_k=A$ and for each $1 \leq i \leq k$ one of the following is true:
    \begin{itemize}
        \item $B_i$ has any of the forms
        \begin{enumerate}
            \item $p \to (q \to p)$
            \item $(p \to (q\to r))\to((p\to q)\to (p \to r))$
            \item $(\lnot p \to \lnot q)\to (q \to p)$,
        \end{enumerate}
        where $p$, $q$ and $r$ are arbitrary formulas. Such a $B_i$ is called an axiom (in the textbook Frege system).
        \item There are $1\leq j_1, j_2 < i$ such that $B_{j_1}=p$, $B_{j_2}=(p\to q)$ and $B_i=q$. Such a $B_i$ is said to be introduced by the \emph{modus ponens} rule: \[\frac{p, p\to q}{q}\]
    \end{itemize}

\end{defi}

\begin{exmp}
    Prove $(a \to a) \to (a \to (a \to a))$ in the textbook Frege system.
\end{exmp}

\begin{exmp}
    Prove $(a\to b) \to (a \to a)$ in the textbook Frege system.
\end{exmp}

\begin{exmp}[Bonus]
    Prove $a \to a$ in the textbook Frege system.
\end{exmp}

\begin{prbl}
    Does every tautology have a polynomial sized proof in the textbook Frege system?
\end{prbl}

\end{document}