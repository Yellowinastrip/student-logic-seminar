\documentclass{article}
\usepackage{babel}[english,czech]
\usepackage{inputenc}[utf8]
\usepackage{amsmath}
\usepackage{amsthm}
\usepackage{amssymb}
\usepackage{indentfirst}

\theoremstyle{definition}
\newtheorem{thrm}{Theorem}
\newtheorem{excs}[thrm]{Exercise}
\newtheorem{exmp}[thrm]{Example}
\newtheorem{prbl}[thrm]{Open problem}
\newtheorem{defi}[thrm]{Definition}
\newtheorem{defip}[thrm]{``Definition"}

\newcommand{\abs}[1]{\lvert #1 \rvert}
\renewcommand{\P}{\textbf{P}}
\newcommand{\NP}{\textbf{NP}}
\newcommand{\coNP}{\textbf{coNP}}
\newcommand{\Land}{\bigwedge}
\newcommand{\Lor}{\bigvee}

\begin{document}
\section*{Formal Proofs and their Lengths II}

\begin{exmp}
    The DNF $(x_1\land y_1)\lor (x_2\land y_2) \lor \dots \lor (x_n \land y_n)$ when expressed as CNF has at least $2^n$ clauses.
\end{exmp}


\subsection*{Propositional Proof Systems}

\begin{defi}
    Let $A$ be a finite set of symbols. We define $A^{\leq n}:= \bigcup_{i=0}^n A^i$ and $A^* := \bigcup_{i\geq 0} A^i$.
\end{defi}

\begin{defi}
    A predicate $f:\{0,1\}^* \to \{0,1\}$ is in $\P$ if there is a Turing machine $M$ computing $f$ in polynomial time\footnote{The precise definition of a Turing machine in fact does not matter. If you have never encountered the definition of a Turing machine, it is enough to consider the intuitive idea of an algorithm, whose number of steps does not exceed a specific polynomial in the length of the input and this itself just means, that the algorithm is somehow feasible --- does not run too long. For example, such an algorithm cannot look at every truth assignment of a formula it receives as an input.}.
\end{defi}

\begin{defi}[Cook-Reckhow] A \emph{propositional proof system} (or a PPS) $P$ is determined by a predicate $f(x,y)$ in $\P$ such that for every propositional formula $A$:
    \begin{itemize}
        \item Soundness:
    \[
     (\exists y\in \{0,1\}^*) \, f(A,y)=1
        \implies
        \text{$A$ is a tautology}, 
     \]
        \item Completeness:
    \[
     (\exists y\in \{0,1\}^*) \, f(A,y)=1
        \impliedby
        \text{$A$ is a tautology}, 
     \]
    \end{itemize}
    here we interpret $f$ to be a predicate checking that $y$ is a valid ``proof'' of $A$. That is, if $f(A,y)=1$, then we say $y$ is a $P$-proof of $A$.
\end{defi}

\begin{exmp}
    The truth-table proof system is a system determined by a predicate     
    \[f(A,y)=\begin{cases}
        1 & \text{$y$ is the truth-table of $A$, $(\forall \overline x) \textbf{tt}_A(\overline x)=1$,}\\
        0 & \text{otherwise.}
    \end{cases}\]
\end{exmp}

\begin{excs}
    Show that the truth-table proof system is a propositional proof system by the definition of Cook-Reckhow.
\end{excs}

\begin{excs}[First lower bound!]
    Show that some truth-table proof of a tautology is exponentially long in the size of that tautology.
\end{excs}


\subsection*{A Little Bit of Complexity}

\begin{defi}
    A predicate $f: \{0,1\}^* \to \{0,1\}$ is in $\NP$ if there is a function $g(x,y)$ in $\P$ and a polynomial $p$ such that for every $x\in \{0,1\}^n$:
    \[ f(x)=1 \iff (\exists y\in\{0,1\}^{\leq p(n)}) \, g(x,y)=1, \]
    if such a $y$ exists it is called the \emph{witness}.
\end{defi}

\begin{defi}
    A predicate $f: \{0,1\}^* \to \{0,1\}$ is in $\coNP$ if there is a function $g(x,y)$ in $\P$ and a polynomial $p$ such that for every $x\in \{0,1\}^n$:
    \[ f(x)=0 \iff (\exists y\in\{0,1\}^{\leq p(n)}) \, g(x,y)=0.\]
\end{defi}

\begin{excs}
    Show that $f(x)\in \NP$ if and only if $\lnot f(x) \in \coNP$. 
\end{excs}

\begin{defi}
    $\text{CNF-SAT}$ is the predicate which assigns $1$ exactly to those CNF formulas which are satisfiable.
    $\text{DNF-TAUT}$ is the predicate which assigns $1$ exactly to those CNF formulas which are satisfiable.
\end{defi}

\begin{thrm}[Cook-Levin]
    The following equalities hold:

    \begin{itemize}
        \item $\P=\NP$ if and only if $\text{CNF-SAT}\in \P$.
        \item $\P=\coNP$ if and only if $\text{DNF-TAUT} \in \P$
        \item $\NP=\coNP$ if and only if $\text{DNF-TAUT} \in \NP$ \\
        if and only if $\text{CNF-SAT}\in \coNP$
    \end{itemize}
\end{thrm}


\begin{thrm}[Cook-Reckhow]
    $\NP=\coNP$ if and only if there is a propositional proof system $P$ which has polynomial sized $P$-proofs of every tautology.
\end{thrm}

\begin{excs}
    Prove the Cook-Reckhow theorem.
\end{excs}

\subsection*{Frege systems I}

\begin{defi}
    The textbook Frege proof system is determined by the proofs of the following form:

    The connectives in every formula in the system are just $\{\lnot, \to \}$. A proof of a formula $A$ is a sequence of propositional formulas $(B_1,\dots, B_k)$, where $B_k=A$ and for each $1 \leq i \leq k$ one of the following is true:
    \begin{itemize}
        \item $B_i$ has any of the forms
        \begin{enumerate}
            \item $p \to (q \to p)$
            \item $(p \to (q\to r))\to((p\to q)\to (p \to r))$
            \item $(\lnot p \to \lnot q)\to (q \to p)$,
        \end{enumerate}
        where $p$, $q$ and $r$ are arbitrary formulas. Such a $B_i$ is called an axiom (in the textbook Frege system).
        \item There are $1\leq j_1, j_2 < i$ such that $B_{j_1}=p$, $B_{j_2}=(p\to q)$ and $B_i=q$. Such a $B_i$ is said to be introduced by the \emph{modus ponens} rule: \[\frac{p, p\to q}{q}\]
    \end{itemize}

\end{defi}

\begin{exmp}
    Prove $(a \to a) \to (a \to (a \to a))$ in the textbook Frege system.
\end{exmp}

\begin{exmp}
    Prove $(a\to b) \to (a \to a)$ in the textbook Frege system.
\end{exmp}

\begin{exmp}
    Prove the textbook Frege system is sound.
\end{exmp}

\begin{exmp}[Bonus]
    Prove $a \to a$ in the textbook Frege system.
\end{exmp}


\begin{prbl}
    Does every tautology have a polynomial sized proof in the textbook Frege system?
\end{prbl}

\end{document}