\documentclass{article}
\usepackage{babel}[english,czech]
\usepackage{inputenc}[utf8]
\usepackage{amsmath}
\usepackage{amsthm}
\usepackage{amssymb}
\usepackage{indentfirst}

\theoremstyle{definition}
\newtheorem{thrm}{Theorem}
\newtheorem{excs}[thrm]{Exercise}
\newtheorem{lemm}[thrm]{Lemma}
\newtheorem{fact}[thrm]{Fact}
\newtheorem{exmp}[thrm]{Example}
\newtheorem{prbl}[thrm]{Open problem}
\newtheorem{defi}[thrm]{Definition}
\newtheorem{defip}[thrm]{``Definition"}

\newcommand{\abs}[1]{\lvert #1 \rvert}
\renewcommand{\P}{\textbf{P}}
\newcommand{\NP}{\textbf{NP}}
\newcommand{\coNP}{\textbf{coNP}}
\newcommand{\Land}{\bigwedge}
\newcommand{\Lor}{\bigvee}
\newcommand{\w}{\textbf{w}}
\renewcommand{\k}{\textbf{k}}

\begin{document}
\section*{Formal Proofs and their Lengths III}

\subsection*{Frege systems I}

\begin{exmp}
    Prove $a \to a$ in the textbook Frege system.
\end{exmp}

\subsection*{A Little Bit of Complexity}

\begin{defi}
    A predicate $f: \{0,1\}^* \to \{0,1\}$ is in $\NP$ if there is a predicate $g(x,y)$ in $\P$ and a polynomial $p$ such that for every $x\in \{0,1\}^n$:
    \[ f(x)=1 \iff (\exists y\in\{0,1\}^{\leq p(n)}) \, g(x,y)=1, \]
    if such a $y$ exists it is called the \emph{witness}.
\end{defi}

\begin{defi}
    A predicate $f: \{0,1\}^* \to \{0,1\}$ is in $\coNP$ if there is a predicate $g(x,y)$ in $\P$ and a polynomial $p$ such that for every $x\in \{0,1\}^n$:
    \[ f(x)=0 \iff (\exists y\in\{0,1\}^{\leq p(n)}) \, g(x,y)=0.\]
\end{defi}

\begin{excs}
    Show that $f(x)\in \NP$ if and only if $\lnot f(x) \in \coNP$. 
\end{excs}

\begin{defi}
    $\text{CNF-SAT}$ is the predicate which assigns $1$ exactly to those CNF formulas which are satisfiable.
    $\text{DNF-TAUT}$ is the predicate which assigns $1$ exactly to those CNF formulas which are satisfiable.
\end{defi}

\begin{thrm}[Cook-Levin]
    The following equalities hold:

    \begin{itemize}
        \item $\P=\NP$ if and only if $\text{CNF-SAT}\in \P$.
        \item $\P=\coNP$ if and only if $\text{DNF-TAUT} \in \P$
        \item $\NP=\coNP$ if and only if $\text{DNF-TAUT} \in \NP$ \\
        if and only if $\text{CNF-SAT}\in \coNP$
    \end{itemize}
\end{thrm}


\begin{thrm}[Cook-Reckhow]
    $\NP=\coNP$ if and only if there is a propositional proof system $P$ which has polynomial sized $P$-proofs of every tautology.
\end{thrm}

\begin{excs}
    Prove the Cook-Reckhow theorem.
\end{excs}

\subsection*{Frege systems II}


\begin{defi}[Frege rule]
    Let $L$ be a complete system of logical connectives. An $\ell$-ary \emph{Frege rule} is a an $(\ell+1)$-tuple of formulas $A_1,\dots, A_l, A_0$ (using just the connectives from $L$, $L$-formulas) written as \[\frac{A_1,\dots,A_l}{A_0},\]
    such that $A_1,\dots, A_n \models A_0$. A $0$-ary Frege rule is called a \emph{Frege axiom scheme}.
\end{defi}

\begin{defi}[Frege proof]
   Let $F$ be a finite set of Frege rules in a finite set of connectives $L$. An $F$-proof of an $L$-formula $C$ from formulas $B_1,\dots,B_t$ is any sequence of formulas $D_1,\dots,D_k$, such that:
   \begin{itemize}
        \item $D_k = C$
        \item For all $i=1,\dots,k$ at least one of the following holds:
        \begin{itemize}
            \item $D_i\in\{B_1,\dots,B_t\}$
            \item There is a Frege rule \[\frac{A_1,\dots,A_\ell}{A_0}\in F, \] and numbers $j_1,\dots,j_\ell < i$ and a substitution\footnote{A mapping from variables to formulas, when applied to a formula it outputs a formula where each variable is replaced by the respective formula according to $\sigma$.} $\sigma$ such that \[\sigma(A_1)=D_{j_1},\dots,\sigma(A_\ell)=D_{j_\ell}\text{, and } \sigma(A_0)=D_i.\] 
        \end{itemize}
   \end{itemize}
   The fact that $\pi$ is an $F$-proof of $C$ from $B_1,\dots,B_t$ is denoted
   \[\pi: B_1,\dots,B_t \vdash_F C, \]
   if we drop the `$\pi:$' part, we just mean that such a $\pi$ exists. 

   We call the number of formulas in a proof $k$ the \emph{number of steps} and denote it $\k(\pi)$. We call the length of the longest formula in $\pi$ the \emph{width} of $\pi$ and denote it $\w(\pi)$. We call the size of $\pi$ the sum of the lengths of all formulas in $\pi$ and denote it $\abs{\pi}$.
\end{defi}

\begin{defi}[Frege system]
A finite set of Frege rules $F$, with formulas using the connectives from a finite complete set $L$, is a \emph{Frege proof system} if it is \emph{sound} (cannot derive a non-tautology) and \emph{implicationally complete} that is: For any $L$-formulas $B_1,\dots,B_t,C$ we have
\[B_1,\dots, B_t \models C \iff B_1,\dots, B_t \vdash_F C.\]
\end{defi}

\begin{fact}
    The textbook Frege system is implicationally complete.
\end{fact}

\begin{excs}
    Show that the textbook Frege system is a Frege system. How many Frege rules does it have?    
\end{excs}

\begin{excs}[Frege can prove substitutions!]
    Show that if $F$ is a Frege system in finite complete set of connectives $L$ and $\pi = (D_1,\dots,D_k)$ fulfills \[\pi:B_1,\dots,B_t \vdash_F C,\]
    and $\sigma$ is a substitution then for some $\pi'$,
    \[\sigma(B_1),\dots,\sigma(B_t) \vdash_F \sigma(C).\]
    What's the smallest $\k(\pi')$ you can achieve?
\end{excs}

\begin{lemm}[Deduction lemma]
    Let $F$ be a Frege system. Assume that \[\pi:A,B_1,\dots, B_t \vdash_F C,\] then there is $\pi'$ such that \[\pi': B_1,\dots, B_t \vdash_F A\to C, \] with $\k(\pi')=O(\k(\pi))$, $\w(\pi')=O(\pi)$ and $\abs{\pi'}\leq O(\abs{\pi}^2)$. 
\end{lemm}

\begin{excs}
    Show that there is a proof of $\lnot \lnot a \to a$ in the textbook Frege system using the Deduction lemma. That is, find a proof: \[\pi:\lnot \lnot a \vdash_{\text{textbook Frege}} a\]
\end{excs}

\begin{fact}
    Let $C$ be a tautology which is not a substitution instance of any shorter tautology and let $F$ be a Frege system. Then any $\pi:\: \vdash_F C$, must have $\k(\pi)=\Omega(ldp(C))$ and $\abs{\pi}=\Omega(m)$, where $ldp(C)$ is the logical depth of $C$ which is defined to be the length of the longest path in the representation tree of $C$ and $m$ is the sum of all lengths of subformulas of $C$. 
\end{fact}

\begin{excs}
    Use the previous fact to prove that any Frege proof \[\pi:\: \vdash_F \overbrace{\lnot \dots \lnot}^{2n}(a\to a),\] must have $\k(\pi)$ at least $\Omega(n)$ and $\abs{\pi}$ at least $\Omega(n^2)$.
\end{excs}

\begin{fact}[Reckhow's Theorem]
   Any two Frege systems $F_1$ and $F_2$ have sizes of their shortest proofs of any particular sequence of tautologies polynomially related. 
\end{fact}

\begin{prbl}
    Let $F$ be a Frege system. Prove any lower bound on the size of $F$-proofs on a sequence aby sequence that is larger than $\Omega(n^2)$.
\end{prbl}

\end{document}